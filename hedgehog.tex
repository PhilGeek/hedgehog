One consequence of the professionalization of philosophy is its ever increasing specialization. Within the normative domain, what has come to be known as metaethical reflection is conducted increasingly independent of substantive ethical reflection. Even if this trend is intelligible given the economic and institutional pressures that spawned it, it is reasonable to wonder if insights are lost and distortions introduced by focusing exclusively on the limited perspective of metaethical reflection. Dworkin laments this trend and is skeptical about contempary metaethics.

The division between "first-order", "substantive" normative inquiry and "second-order", "meta" normative inquiry has come to seem natural to us and is fundamental to the way we standardly present these topics in our teacing. But it has not always been so. Thus, for example, John Rawls has claimed that the moral philosophies of Hume and Kant cannot intelligibly be presented in this way. Instead, they exmplify what he calls a "philosophical ethics". [Characterization of philosophical ethics] 

Dworkin, in the first part of "Justice for Hedgehogs", presents a distinct alternative to the prevailing orthodoxy. All second-order, meta-normative claims are to be understood, fundamentally, as first-order, substantive, normative claims. If true, then meta-normative inquiry, or what passes for it, could not intelligibly be conducted indepenently of substantive, normative reflection. If Dworkin is right, then contemporary metaethics rests upon a mistake. (As will emerge, this echo of Prichard is deliberate.)

Dworkin's brief against metaethics is part of a larger case for the unity of value. The unity of value is the great idea of the book in virtue of which Dworkin counts as a hedgehog set against the prevailing orthodoxy of foxes. According to the unity of value, apparently distinct values, such as liberty and equality, are not utterly distinct and so the reasons they give rise to could not practically conflict. What it is possess liberty presupposes the value of equality: "You cannot determine what liberty requires without also deciding what distribution of property and opportunity shows equal concern for all." (4) But if that is right, then the demands of liberty could not practically conflict with the demands of equality. In this way, the advocate of the unity of value will resist the Pyrhhonian skepticism that proceeds from an argument from conflicting values. (For a useful to discussion of such argument forms see Annas and Barnes) As thus far presented, the unity of value is a complex philosophical doctrine, but, as should be clear from the form that Dworkin's skepticism about metethics takes, and as Dworkin himself insists, it is also a creed.

An aspect of the overall case for the unity of value can potentially shed light on Dworkin's skeptical attitude towards metaethics. fiSpecifically, Dworkin's views about the nature of interpretation and the part they play in the argument for the unity of value are relevant to his skepticism about metaethical inquiry.