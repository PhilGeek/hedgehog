%!TEX TS-program = xelatex
%!TEX TS-options = -synctex=1
%!TEX encoding = UTF-8 Unicode
%
%  geach
%
%  Created by Mark Eli Kalderon on 2011-12-26.
%  Copyright (c) 2011. All rights reserved.
%

\documentclass[12pt]{article} 

% Definitions
\newcommand\mykeywords{Dworkin, metaethics} 
\newcommand\myauthor{Mark Eli Kalderon} 
\newcommand\mytitle{The Truth about Morals}

% Packages
\usepackage{geometry} \geometry{a4paper} 
\usepackage{url}
\usepackage{txfonts}
\usepackage{color}
\definecolor{gray}{rgb}{0.459,0.438,0.471}
% \usepackage{setspace}
% \doublespace % Uncomment for doublespacing if necessary
% \usepackage{epigraph} % optional

% XeTeX
\usepackage{fontspec}
\usepackage{xltxtra,xunicode}
\defaultfontfeatures{Scale=MatchLowercase,Mapping=tex-text}
\setmainfont{Hoefler Text}
\setsansfont{Gill Sans}
\setmonofont{Inconsolata}

% Bibliography
\usepackage[round]{natbib} 

% Title Information
\title{\mytitle} % For thanks comment this line and uncomment the line below
% \title{\mytitle\thanks{}}% 
\author{\myauthor} 
\date{} % Leave blank for no date, comment out for most recent date

% PDF Stuff
\usepackage[plainpages=false, pdfpagelabels, bookmarksnumbered, backref, pdftitle={\mytitle}, pagebackref, pdfauthor={\myauthor}, pdfkeywords={\mykeywords}, xetex, colorlinks=true, citecolor=gray, linkcolor=gray, urlcolor=gray]{hyperref} 

%%% BEGIN DOCUMENT
\begin{document}

% Title Page
\maketitle
 
\vskip 2em \hrule height 0.4pt \vskip 2em
% Main Content

% Layout Settings
\setlength{\parindent}{1em}


One consequence of the professionalization of philosophy is its ever increasing specialization. Within the normative domain, what has come to be known as metaethical reflection is increasingly conducted independently of substantive ethical reflection. Even if this trend is intelligible given the economic and institutional pressures that spawned it, it is reasonable to wonder if insights are lost and distortions introduced by focusing exclusively on the limited perspective of metaethical inquiry. Dworkin laments this trend and is skeptical about contempary metaethics.

The division between ``first-order'', ``substantive'' normative inquiry and ``second-order'', ``meta'' normative inquiry has come to seem natural to us and is fundamental to the way we standardly present these topics in our teacing. But it has not always been so. Thus, for example, \citet{Rawls:2000uq} has claimed that the moral philosophies of Hume and Kant cannot intelligibly be presented in this way. Instead, they exemplify what he calls a ``philosophical ethics''. [Characterization of philosophical ethics] 

Dworkin, in the first part of \emph{Justice for Hedgehogs}, presents a distinct alternative to the prevailing orthodoxy. All second-order, meta-normative claims are to be understood, fundamentally, as first-order, substantive, normative claims. If true, then meta-normative inquiry, or what passes for it, could not intelligibly be conducted independently of substantive, normative reflection. If Dworkin is right, then contemporary metaethics rests upon a mistake. (As will emerge, this echo of \citealt{Prichard:1912zm} is deliberate.)

Dworkin's brief against metaethics is a small part of a larger case for the unity of value. The unity of value is the great idea of the book in virtue of which Dworkin counts as a hedgehog set against the prevailing orthodoxy of foxes. The unity of value is not the claim that there exists one master value to which all other values must reduce; rather, it is a distinct value monism that claims that all values are mutually dependent. What it is to be a value of a certain kind will depend upon what it is to be a value of a distinct kind. Consider the following analogy. \citet{Schaffer:2007ma,Schaffer:2008ks,Schaffer:2009vn,Schaffer:2010ja} distinguishes between existence monism, according to which only one thing exists---the world, and priority monism. According to priority monism, the world may contain a plurality of parts, but these parts depend upon the whole whose parts they are. Thus whereas Parmenides is an existence monist---there exists only the one being of the way of truth, Hegel is a priority monist. The unity of value is more akin to priority monism than to existence monism. The analogue of existence monism would be the claim that there exists only one value to which all other values must reduce. In contrast, the unity of value can allow for a plurality of values, it is just that these values are not independent of one another but are rather mutually dependent.

According to the unity of value, distinct values, such as liberty and equality, mutually depend upon one another in such a manner that the reasons they give rise to could not practically conflict. What it is possess liberty presupposes the value of equality: ``You cannot determine what liberty requires without also deciding what distribution of property and opportunity shows equal concern for all'' \citep[4]{Dworkin:2011fk}. But if that is right, then the demands of liberty could not practically conflict with the demands of equality. In this way, the advocate of the unity of value will resist the Pyrhhonian skepticism, revived in modern times by Montaigne, that proceeds from the argument from conflicting values \citep[for a useful to discussion of such argument forms see][]{Annas:1985fk}. As thus far presented, the unity of value is a complex philosophical doctrine, but, as should be clear from the form that Dworkin's skepticism about metaethics takes, and as Dworkin himself insists, it is also a \emph{creed}.

An aspect of the overall case for the unity of value sheds light on Dworkin's skeptical attitude towards metaethics. Specifically, Dworkin's views about the nature of interpretation and the part they play in the argument for the unity of value are relevant to his skepticism about metaethical inquiry. Interpretation, for Dworkin, is an important mode of understanding that finds application in such diverse areas as literary criticism, history, and law. Despite the diversity of application, this mode of understanding nevertheless displays a common structure. \citet[130--134]{Dworkin:2011fk} provides an account of this common structure in his \emph{value theory} of interpretation. According to the value theory of interpretation, there are different genres of interpretation. To interpret something one must first assign it to a particular genre of interpretation. A given genre of interpretation is governed in part by a value or range of values. Thus, for example, in interpreting a poem one may be primarily interested in poetic beauty, say. Thus to interpret something one must second identify the value or values that govern the given genre of interpretation. Finally, one must provide an interpretation that best realizes the governing value or values of the given genre of interpretation. 

Interpretation, on the value theory, is value laden. It is also pervasive---it is a mode of understanding deployed, \emph{inter alia}, in our daily commerce with our fellows. If interpretation is value-laden and pervasive, then since it is an inherently truth-seeking activity, values will themselves be pervasive. Moreover, the values posited by our best interpretations will display a kind of unity. Interpretation is inherently holistic, its methods coherentist. The best interpretation of liberty will make it cohere with equality, and indeed with other important political values. And since interpretation is a truth-seeking activity in virtue of which it counts as a genuine mode of understanding, pervasive values must display the unity required if they are to so cohere.


% Bibligography
\bibliographystyle{plainnat} 
\bibliography{Philosophy} 

\end{document}